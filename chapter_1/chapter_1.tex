\chapter{Introduction}
\label{chp:intro}

\graphicspath{{chapter_1/figures/}} % path to the figures folder of this chapter

%% ---------------------------------------------------------------
% The following annotation is customary for chapter which have already been published as a paper:
%\blfootnote{Parts of this chapter have been published in Computer Methods in Applied Mechanics and Engineering \textbf{320}, 633 (2017) \cite{bessa2017a}.}

%\authors{Miguel {\titleshape Bessa}} % Only include authors in the rare case when multiple people contributed significantly to this chapter.

%% In case you want to have an Epigraph, uncomment the next 4 lines:
%\epigraph{
%    Nature and nature's laws lay hid in the night; \\
%    God said `Let Newton be!' and all was light.
%}{Alexander Pope}

%% In case you want to include an abstract for the chapter, uncomment the next lines:
%\begin{abstract}
%	Chapter abstract here.
%\end{abstract}

%% You may decide to start the actual chapter on a new page. If so, uncomment the next line:
\newpage
%% ---------------------------------------------------------------

%\dropcap{T}{he} introduction chapter should be short (1 to 3 pages). In the first paragraph, briefly refer the intent of this work and the main solution proposed. Then use the following paragraphs to provide the Big Picture\footnote{In other words: Why should we care? Including one figure motivating the thesis can be useful.}, and discuss the main challenge(s) without presenting alternative solutions (that's for Chapter \ref{chp:lit_rev}). The last paragraph typically contains the thesis' structure.

%\begin{itemize}
	%\item Chapter \ref{chp:lit_rev} provides some guidelines for the literature review.
	%\item Chapter \ref{chp:intro_to_latex} briefly introduces Latex
	%\item Chapter \ref{chp:basic_software} lists the important Software.
%\end{itemize}


The Ministry of Defence (MoD) is trying to keep the equipment of it's forces at full capability through intensive maintenance at home bases and in remote areas. The systems in use (vehicles, weapon systems, auxiliary systems etc.) can have different sets of complexity and date from different era's. At home bases, these are often replaced by of the shelf parts that are expected to fail over a certain time. However, it is often difficult to predict the lifetime of these parts which poses a logistical challenge. Additionally, these parts are expensive due to their infrequent production. Also, many systems are outdated or have been produced by a manufacturer that stopped production of a certain part. In these cases it pays off to have a in-house production capacity to be self-sufficient in a range of parts that are often or in-frequently needed and are either difficult to come by, expensive or have a long logistical footprint. For these cases the MoD has a limited production capacity of different machinery to produce a range of parts. Also, for the modifications of systems (upgrades and iterations), self produced or enhanced parts can be produced on base. The above described situation is even more crucial and complicated in foreign operation bases or on remote operating platforms such as ships. A rare part is difficult to come by in a remote area and can be extremely time consuming and expensive to obtain. 

The employment of 3D printers (additive manufacturing or rapid manufacturing) can help reduce the above mentioned issues. In the past decades (1980-2010) 3D printing has evolved to a production method that starts to gain respect in the engineering and manufacturing environment. In the last years the process, materials and product of 3D printers have been thoroughly investigated. Currently the polymer fused filament fabrication (FFF) dominates the market due to it's simplicity and low cost. There are different systems with a wide range of materials available, from metals to ceramics, but these processes are often complex, expensive and require have machinery. According to Wohlers Report in 2017 \cite{WohlersAssociates2017WohlersIndustry} the majority of 3D printing services focus on polymer systems and materials. Polymer based parts are a huge part of our current society, in consumer products as well as in engineering systems. The relevance of 3D printing polymer parts for Defence applications have been studied thoroughly in different analysis and studies \cite{Bastiaans2015DeDefensie} \cite{NATOPerspectivesOperations} \cite{Joyce20143DDefense}, these explain the different significant benefits for the MoD: smaller logistic chains, possibility to produce cheaper alternatives with respect to conventional manufactures, fast innovation and iterations of products, production of complex geometries, in-house development of (classified) systems, the production of parts for obsolete systems due to unavailable producers, and finally the production of parts in remote areas.

These arguments are the inducements for the MoD to start exploring and implementing the possibilities of 3D printing. At the beginning of the 2010's different departments inside the MoD started to acquire simple systems to produce and experiment with early prototypes and non-critical parts. This lead to the acquisition of more complex FFF and additional machines to produce functional parts. Despite these being of low quality and for non-critical parts, 3D printing seemed to fill in a need for small complex parts that normally took too long to order or to make. Also, due to the short lead time a large wave of iterations and innovations came from these small 3D printing groups. Currently, 3D printers are being implemented in different departments at home bases and abroad, first pilots at operation bases and on ships have been successful and are being implemented as permanent tools for fast production of non-critical parts. At this point, considering the success of the FFF implemented systems, the MoD wants to expand it's knowledge and production capacity towards more critical parts. The logical choices being investing in metal additive manufacturing systems or optimizing the results of the FFF system. Despite that the FFF process has been optimized to a large degree, the results are still not comparable to conventional produced polymers. Often the mechanical properties of FFF printed polymer parts are as bad as 50\% of e.g. injection moulded parts. This significantly obstructs the MoD from implementing FFF products in functional appliances. The first step towards the wider use of FFF parts in the logistical chain would be to identify and predict its mechanical behaviour and quality.

To be able to produce and predict the properties of FFF printed polymers more knowledge needs to be acquired before the MoD can implement FFF for critical polymer parts. TNO (Nederlandse Organisatie voor Toegepast Natuurwetenschappelijk Onderzoek) is one of the advisers and research organizations that advises the MoD on related subjects. One of the key factors in increasing the mechanical properties of FFF printed parts is understanding the physical phenomena of the process and the materials science of the produced parts, which is the subject of this thesis. The goal of this thesis is to identify the mechanical behaviour, and to predict it trough the modification of known models and simulation tools.

In the first part of the thesis, an extensive literature study of the FFF principles and the available research will be discussed. Based on the literature the knowlegde gap will be identified and a hypothesis and related research questions are presented.

The second part of the thesis consists of four chapters in which a determined part of research is presented and discussed, each with its methodologies, results, discussions and conclusions. Chapter 4 will discuss the investigation on the mesostructure and the definition of a new Representative Volume Element. Chapter 5 present the empirical testing of FFF products and the complications involved with it. Chapter 6 discusses the elastic properties of the Representative Volume Element, and applies a Finite Element Method homogenization approach to determine the elasticity in the 3 principle directions. Chapter 7 is discusses the most important knowledge gap, the prediction of non-linear behaviour of FFF product through the Finite Elemnet Method. In this chapter the effort and information of the previous chapters is combined to substantiate the effectiveness of the model and  the results. 
Finally, a global conclusion and recommendation will be presented in the last chapter. 



%This document will first discuss the known research involving the FFF process en products, and additionally the knowledge on models and simulations that can be implemented for FFF. This includes the use of theories involving composites, homogenization and the use of FEM software like Abaqus. Afterwards, experiments on the characterization of mechanical properties and the identification of the phenomena are presented. The following chapters are concerned with finite element analysis (FEA) of the mechanical properties and the comparison with other analytical methods. The results will finally be discussed and compared with the experimental data. Based on this comparison, a discussion and conclusion will follow, where the difference and recommendations are presented. 









