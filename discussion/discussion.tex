\chapter{Discussion}
\label{chp:discussion}

\dropcap{H}{aving} a chapter for the discussion of the results is not mandatory, but it is usually very useful. Again, unfortunately most people will not read the entire thesis\footnote{But do not worry: I will.}. They will read the abstract (summary), see the figures and then read the Discussion and Conclusions. Only if the findings are worthwhile, then they may read part or all of the content.

Therefore, you want to devote a lot of time to the parts that are most important: Abstract, Discussion and Conclusions. Of these 3, the ``Discussion'' is perhaps the most difficult because you need to demonstrate that you have good critical reasoning about the work you did:

\begin{itemize}
	\item What is the general impact of your work?
	\item What do the results mean?
	\item Compare your results with the literature (in a concise manner, and only if applicable).
	\item Be specific about the improvements you made (whether there is prior literature or not), but also discuss the limitations of your work.
	\item Highlight unexpected results (if any).
\end{itemize}

You need to be short but extremely on point. One strategy is to start general (to increase the relevance of your work), but then to be very specific in order to demonstrate that your work is truly useful to solve a particular problem.
