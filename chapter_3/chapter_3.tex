\chapter{Your innovative chapter}
\label{chp:innovative_chapter}

\graphicspath{{chapter_3/figures/}} % path to the figures folder of this chapter

\dropcap{T}{his} is the Chapter where the magic starts! Here you have to find a balance between providing enough detail and not disrupting the storyline of your work. This is usually achieved by a combination of two points:

\begin{itemize}
	\item Strive for a concise writing style (write a first draft and then start shortening it progressively)
	\item In the chapters include only key figures and tables. Think about what is essential to understand your work
	\item Use the Appendix! You can include many details and additional results in the appendix. See Appendix \ref{app:appendix_a} for a few more comments on this.
\end{itemize}

\section{Composite model}
For the prediction and homogenization of FFF RVE's a model proposed by Miguel Bessa \cite{bessa2017a} is adjusted for the use of of FFF RVE's. Intentionally this RVE was made for long fieber reinforced plastics, as has been discussed in \ref{chp:lit_rev} FFF parts can be treated as long fiber parts. Therefor, this is a similar approach as Rodriguez \cite{Rodriguez2003MechanicalModeling} and Somireddy \cite{Somireddy2018DevelopmentFDM} use. 