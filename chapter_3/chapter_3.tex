\chapter{Hypothesis and research questions}
\label{chp:hypothesis}

\graphicspath{chapter_3_hypothesis/figures/} % path to the figures folder of this chapter

\section{Knowledge gap in the current state of the art}
\label{Knowledge gap in the current state of the art}
The current use and experimentation in the field of FFF tries to bridge the gap between simple consumer parts towards functional and mechanically reliable parts. For this a significant amount of effort has been put in the characterization of mechanical properties through loading tests, including the effect that process parameters have on the samples. It has been concluded that the properties are dramatically different from bulk material due to porosity and weak bonding areas. To be able to predict these anisotropic properties, different approaches have been proposed by multiple researchers, with considerable effort from  Rodriguez et al \cite{Rodriguez2003MechanicalModeling}. 

To determine the effect of porosity on an FFF product the mesostructure is first analysed using microtoming. The results from different systems can significantly alter in mesostructure, where the quality and accuracy of the mechatronics are most likely the cause of this discrepancy. Due to the different observations no clear agreement was concluded between the researchers on the formation of this porosity. The Ultimaker S5 is a current FFF systems that shows promise to achieve reproducible quality.

The empirical loading tests conducted by different researchers often used testing standards for isotropic polymers. The geometry of the samples used resulted in stress concentrations and crack initiation points, causing the samples to fail prematurely. Additionally, the production procedure is not yet standardized, meaning that the produced samples show significant difference between research methodologies.

There have been several attempts to analyze and predict the degree of bonding of the road interface. Since this is a complex time dependent parameter, no exact prediction was made to this date on the bonding between roads. Most researchers assumed perfect healing (bulk properties) of the interface. 

Multiple researchers found overlap between FFF products and composites, therefore they applied elasticity models such as the mechanics of materials approach and the Classical Laminate Theory to FFF products. Additionally, a few Finite Element Analysis where performed to determine the elastic properties of FFF products. The results of the elastic models are in accordance with the experimental results.

Since the non-linear behaviour of polymers is complex and difficult to predict, due to their molecular structure and strain rate dependency, the FEM knowledge on this topic is limited. For composites that incorporate an epoxy and long fibres, some validated models have been proposed on a micro-mechanical scale. Epoxies have a more predictive non-linear behaviour due to their cross-links. If the understanding of polymer molecular behaviour is sufficient, one could try to adjust the composite model to simulate the micro mechanical properties of FFF products. 

%The issue is that different FFF systems exhibit significant difference in produced quality.
\section{Hypothesis}
Based upon the knowledge gap defined in previous section the following hypothesis has been formulated:

\todo{I updated the hypothesis}

\begin{itemize}
	\item High-fidelity FFF RVEs can be developed to predict non only the elastic beahviour, but also the complete non-linear elasto-plastic, damage and fracture behavior based on mesostructure characterization and elementary experimental tests of the base filament.
\end{itemize}


%"The characteristics involving the degree of bonding and the porosity are the main contributors to the decrease of mechanical properties with respect to the bulk material, these characteristics are strongly influenced by the process parameters and hardware used. By characterizing these properties for a particular combination of material, hardware and software, models could be made that makes a valid prediction of the mechanical properties of produced parts."

\section{Research questions}
    \label{Research questions}
    %NALEZEN!!
Verification of the above mentioned hypothesis involves addressing different research questions:

\todo{I simplified the questions and made them more specific. Please check.}

\begin{enumerate}
	\item What are the processes parameters that influence the mechanical properties the most?
	\item How does the mesostructure obtained from the system in use (Ultimaker S5) differ from the literature, and can a geometric model for this porosity be defined?
	\item What are the elementary tests of the base filament that allow to predict the mechanical behavior of the FFF RVE via FEA?
	\item What experimental tests of the FFF specimens appropriately validate the FEA of the RVE? 
\end{enumerate}

%"What composite or mechanical behaviour theory can be applied to predict the behaviour of FFF products?

%--------------------------------------------------------------
%"What are the process parameters influencing the characteristics,?"

%"What are the optimal process parameters for the Ultimaker S5 system and what are the related mechanical properties?"

%"Is there a clear geometry for the mesotructure of these FFF products, and can the formation of this structure be derived?"

%"What is the optimal test and production procedure for the testing of FFF samples, and what are the related mechanical properties of the FFF samples and filament?

%"Can the elastic and non-linear properties be predicted with a composite RVE micro-mechanical mode and how does this relate to other composite models?"

%"What is the influence of the negative airgap on the porosity?"

%"What is the minimum amount of porosity that is obtainable, without significantly distorting and overdimensioning the product?"

%"What model would give most realistic results predicting the mechanical properties (considering the influence of degree of bonding and porosity)?"

%"Can this model be implemented to predict the properties of complex parts with different layer orientations, different infill, material, process parameters and systems?".

%"How does this model compare to the models from literature and Digimat?

\section{Research methodology}
To answer the research questions stated in the previous section a research methodology needs to be defined. This section will describe the steps necessary to answer the research questions. 

First the system under analysis, Ultimaker S5 (UMS5), is compared with results available in the literature to identify the process parameters that are influencing the mechanical properties. 

Subsequently, the mesostructure is analyzed with micrographs to identify the form and amount of porosity in the UMS5 products. Then, the mesostructure is reproduced digitally and an RVE is defined for further analysis via the FEM. Ideally, this analysis is dependent of the process parameters. 

The ABS filament is then tested to determine the bulk properties of the material. Hence, the FEA of the RVE can be conducted considering the mesostructure and the elasto-plastic and damage models. These models need to be adequate for polymers used in FFF, so it is important to implement the correct constitutive equations in the RVE model. Besides the RVE model, other analytical models can be compared to determine it's effectiveness.  

Finally, it is important to validate the developed RVE model with experimental tests of the FFF specimens and, perhaps include a comparison with other analytical models. The RVE model can afterwards be validated for different process parameters when additional process parameters are empirically tested. Note that different experimental tests may need to be considered for the FFF specimens because boundary effects that cause premature failure should not occur (otherwise, the RVE cannot be used to predict the mechanical behaviour).


%The research methodology is based on the enhancement model that is based on a composite RVE model. To obtain data that can be implemented in the model and data that is needed to verify and validate the model and create iterations, empirical tests are done alongside the development of the model. 

%The research mehtodology is based on the research questions stated in section \ref{Research questions}, the methodology presented aims to define the steps necessary to answer the research questions and subsequently verify or reject the proposed hypothesis. 
%First the results of the Ultimaker S5 systems should be analyzed to compare them with the results of the literature, and determine if the focus of the model should be configured. This includes an analysis on the porosity and mechanical properties. If time allows, an analysis on the thermal history should also be done to get a better view on the degree of bonding of the layers.
%Subsequently, the best parameters should be defined for producing FFF parts with the Ultimaker S5. The effect of most parameters are known, except for the increase in flow of the parts to decrease the prososity of the part. Literature has proven that it can significantly reduce the porosity for other systems, it would be important to know if this would also stand for the Ultimaker S5 system.
%Simultaneously, the dimension of the part should be examined to determine if no unacceptable distortion takes place due to overextrustion.
%When the best parameters are defined, the information generated on the porosity can be used to enhance the model, the model can then be verified and validated according to the mechanical properties. The implementation of cohesive elements/zones might be needed to simulate the best results.
%Furthermore, the model can be used to simulate different scenarios, e.g. $yx[-45/45]_n$, to validate the different cases mechanical testing should be done.
%Additionally, the software used in Digimat should be compared to the findings from the produced model. The software in Digimat applies a plasticity model that could also be introduced in the produced model, however, the benefits from this might be insignificant due to the small plastic response from FFF produced parts. Better would be to generate an additional strength model to determine the UTS and yield strength.
%Finally, the models from the literature, Digimat and mechanical experiments should be compared to the produced model to validate and verify it. Additionally, different cases could be investigated to analyze if these also hold for the model.




