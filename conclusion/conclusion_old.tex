\chapter{Conclusion and recommendations}
\label{chp:8}

\todo{I think your thesis is good, but the Conclusion and Recommendations Chapter needs to improve significantly. It would be a pity if you don't work on this chapter more.}

\todo{A few pointers: 1) relate to the hypothesis and research questions. That will make it much easier to follow; 2) structure your conclusions better, perhaps with bullet points or at least short paragraphs; 3) discuss limitations of your models, for example, you should mention the cohesive elements to model debonding and how much you think this would affect the results; 4) create a section with recommendations: what can you recommend to TNO?;. Please dedicate a few hours to this. It will pay off.}

The current literature and efforts in characterizing the mechanical properties of FFF products have been limited to experimental investigations and predictions on elasticity. A lack of knowledge on the process and a large variation in quality between systems lead to a significant divergence in the use of standards, process parameters and research directions. By researching the fundamental questions on the process and the effect on the mechanical behaviour we were able to identify important process parameters, and the causes of the degradation in mechanical properties. FFF products show significant anisotropy in the 3 principal directions, this is due to the lamina like build up of roads that induce bad healing between layers and most importantly, a large amount (almost 3\%) of periodic porosity. The process parameters that largely affect this are: the aspect ratio between the layer height and the line width, the flow multiplier, the layer orientations, the envelope temperature, the deposition speed and the overall accuracy of the system.

In this thesis research was conducted on ABS filament on a Ultimaker S5 system, one of the most advanced systems in its category available to this date. Trough investigating the porosity in the mesostructure of the FFF product generated by the Ultimaker S5, it could be seen that the mesotructure has its differences and similarities in comparison with the results presented in by other researchers. The mesostructure showed a clear periodicity, with triangle like cavities. When the slicing process was investigated further, it became clear that the printing strategy involved overlapping of roads to minimize porosity. This was validated by micrographs of multi-coloured mesostructures. From this analysis, an RVE was developed based on the overlap theory. The geometry of the RVE was made dependent on the line width and layer height determined by the slicer. 

Simultaneously, empirical tensile tests were performed in the 3 principal direction for validation of the RVE simulations. Unfortunately, no testing or production standards were available for testing FFF products, therefore 2 different methods were used, the ISO 527 and ASTM D 3039, both applied in literature. Both standards had their flaws and the results did not indicate a favourite.  
By using an homogenization approach combined with FEM based on the defined RVE, the elastic properties in the 3 principle directions could be predicted within roughly 5\% accuracy. The Rule of Mixtures predicted the elastic properties with less accuracy in the 2 and 3 direction, since the cross sections are not constant. 

Through an intensive study on the non-linear behaviour of polymers, good stress based yield and strain based failure models were implemented in a explicit analysis where the response to loading conditions in the three principle directions have been investigated. After optimizing the FEM model, the simulations returned results that had significant overlap with the ASTM D 3039 standard tests in each direction, showing a deviation of roughly 5\% in the 1 and 2 directions. Despite the simulated results already achieving high accuracy, a damage model could be implemented based on crazing criteria and strain rates to simulate polymer behaviour even better. The FEM model could be optimized even further by finishing the mesh convergence study with a set of finer meshes, this is however, due to the computationally expensive procedure, difficult. The difference in the 3 direction is explained through the sub-optimal healing between roads. This seemed to have less effect on the mechanical properties than first was expected. With the currently used process parameters, the sub-optimal healing caused a decrease of 10\% relative to the bulk UTS, the rest of the decrease in mechanical properties is provoked by the presence of the periodic porosity. To strengthen this statement, more validation is needed for different aspect ratios using the ASTM D 3039 standard (or a variation).  Subsequently these results can be used to validate the relation with Linear Elastic Fracture Mechanics. If an equation can be fitted for the fracture stress dependent on the the aspect ratio, an analytical approach (which is less time consuming) might suffice for the prediction of the fracture stress. 
When the RVE model is validated for certain aspect ratios, combined RVE's can be modeled to simulate the behaviour of multiple layers with different directions. Since layers in different orientations might limit stress concentrations and unstable crack growth, a potential optimum might be found. This search could benefit from the use of machine learning, through promoting factors that contribute to optimal mechanical behaviour. Additionally, new materials can be introduced to the RVE, along with composite structures with short or long fibre reinforced material.  

Ultimately, a new element could be defined with the properties of optimal process parameters obtained from the RVE. The properties of this element could be used for general FEM simulations for parts in engineering applications. When these properties are known, even topology optimization can be applied to find the optimal shape for the loading state of the body. 
When this is achieved, the FFF will truly manifest its potential trough its large geometrical freedom, creating parts on location that perfectly fit the need of the application with minimal material used. 

\todo{By the way: in the text you refer to a few Appendices. Where are they?}
