\chapter{Conclusion and recommendations}
\label{chp:8}

\section{Conclusion}
%\todo{I think your thesis is good, but the Conclusion and Recommendations Chapter needs to improve significantly. It would be a pity if you don't work on this chapter more.}
%
%\todo{A few pointers: 1) relate to the hypothesis and research questions. That will make it much easier to follow; 2) structure your conclusions better, perhaps with bullet points or at least short paragraphs; 3) discuss limitations of your models, for example, you should mention the cohesive elements to model debonding and how much you think this would affect the results; 4) create a section with recommendations: what can you recommend to TNO?;. Please dedicate a few hours to this. It will pay off.}

\subsection{Identification of process parameters influencing the mechanical properties (research question 1)}
The drop in mechanical properties is, according to literature, accredited to the inherent porosity that is induced during the process and the sub-optimal healing between roads. 

\subsubsection{Porosity} 
The porosity is influenced by the temperature dependant viscosity of the filament, printing strategy, accuracy, quality of the system and most importantly, the aspect ratio, layer height over line width ($l_h/l_w$). 

\subsubsection{Healing} 
Healing is influenced by the filament properties glass transition temperature ($T_g$) reptation time ($t_{rep}$) and entanglement density ($v_e$). The temperature history defines the degree of bonding between adjacent roads. This effect is more articulated in the $z$ direction, due to the larger time period that the toolhead needs to reach the subsequent point in the next layer, giving the lower road sufficient time to cool down. 

\subsubsection{Interroad strength} 
Due to these two phenomena, the interroad strength is drastically reduced, resulting in a brittle failure when loaded. This effect can be reduced to some degree by alternating layer directions in a composite like lamina to distribute the weak links and to achieve more isotropy. 
%These effects are complex and alter with different process parameters and materials. 
Quantifying the effects and optimizing the mechanical properties has only been achieved to a limited extent in current literature. 
%By investigating samples with unidirectional roads, one could get a better understanding on the an-istropical behviour in the 3 principal directions. 
\subsubsection{Elasticity} 
According to theory, only the mesostructure should significantly alter the elastic properties of the FFF products. When the elastic properties are measured with the RVE homogenization approach, a prediction of the elastic properties with an accuracy of approximately 10\% can be made compared to the empirical results.The largest uncertainly for validation this prediction is related to the difficulties regarding the dimensional accuracy of the sample.

\subsection{Mesostructure (research question 2)}
%The mesostructure defines the inherent porosity of FFF products. By investigating the cross sectional area of unidirectional printed parts the amount and geometry can be defined. Since this mesostructure is dependant on the effects stated above, literature shows a wide variation of porosity types. The main characteristic is a void occurring in between the sintering of 4 roads, as is explained in chapter \ref{chp:4}. To obtain consistent results, a single high end system was used (Ultimaker S5) with fixed process parameters to produce FFF samples and investigate these. Through investigating the slicer software (software that defines some process parameters), a derivation of the formation of the mesostructure could be made as can be seen in chapter \ref{chp:3}. 
The mesostructure was constructed with the assumption of two overlapping stadium shaped roads, generating a triangular periodic cavity. The size and geometry of this cavity seemed to be dependent of the aspect ratio and could be defined trough a geometric model. An experimentally-validated finite element analysis of Representative Volume Elements of the mesostructure confirmed this dependence. Not only does this model match the observed mesostructure, but also the results from the FEA indicate a very strong overlap with the empirical results from the ASTM D 3039 testing method (deviation of less than 5\% of the UTS between the fractions relative to bulk material in the relevant directions). 

\subsection{Experimental testing (research question 3)}
%The experimental tensile testing of FFF products seemed problematic due to different factors related to the FFF process.
\subsubsection{ASTM D 3039 and ISO 527 testing}
Stress concentrations and crack initiation points give an unexpected and inconsistent response of the tested samples. The large surface roughness and limited  dimensional accuracy make it difficult to accurately measure the cross sectional area. Between the two tested methods, the ASTM D 3039 seemed to exhibit most representative behaviour, due to the possibility of capturing the true behaviour of the weakened bonds and the effect of the porosity. This statement is strongly supported by the results from the FEA, and the comparison between the responses of the ASTM D 3039 and ISO 527 standards. 

\subsubsection{Filament testing}
Filament testing results seemed indicative of the bulk properties of the ABS filament and showed the same behaviour as injection moulded counterparts as well as filaments tested with appropriate clamps from the literature. Therefore, the results were accepted with this provisional method and introduced into the RVE model as hardening curves.

\subsection{Dominant phenomenon of the decrease in mechanical properties (research question 4)}
With the RVE simulation, the properties of FFF products with fully healed interfaces were simulated since bulk material properties were assumed in the generation of the RVE. This makes it more difficult to compare and validate correctly with the sub-optimally healed samples used for the experimental testing. This limitation introduces a significant uncertainty when comparing it with the empirical results. 
\subsubsection{Drop in mechanical properties in the 3 principal directions}
The results indicate that the simulated directions 1 and 2 have overestimated the UTS for less than 12\% with respect of the ASTM standard. The simulated direction 3 however, has a much higher UTS (43\%) than the experimentally obtained 3 direction. This might be related to the decrease in healing in the $z$ direction.

%This difference exactly correlates with the difference from results obtained by van Veen \cite{Veen2019EnhancingTemperature} by comparing empirically tested samples of sub-optimally healed samples at room temperature. This implies that the difference in UTS between the simulated (fully-healed) RVE, and the empirically tested (sub-optimally healed) samples indeed results from sub-optimal healing. 
Therefore, for the parameters used at room temperature and assuming sub-optimal healing for the direction 1, the drop in UTS is explained by the porosity and accounts for a drop of 10\% with respect to the bulk properties. The direction 2 experiences a drop in UTS of approximately 70\% and is largely accounted by the porosity as well. However, direction 3 experiences the most significant drop; 77\%, from which 10\% is responsible for the healing. This effect could be simulated by adding cohesive elements that simulate a weaker interface between roads. This was not explored in this thesis.

Nevertheless, considering these process parameters, decreasing the porosity is predicted to have more potential of increasing the mechanical properties than increasing the healing between roads. 
%However, this might be much harder than increasing the healing, since the only method currently known to decrease the porosity is by over extruding, generating distortion and dimensional inaccuracies. 
%Additionally, through elevating the temperature even more in the heated envelope, to the point of minimal material diffusion, the crack tips might be blunted, making crack initiation more difficult. This is however, fully hypothetical and should be investigated further. 
\subsubsection{Aspect ratio}
Before being able to conclude that the RVE model is also validated for different aspect ratios (up to this point only the results of aspect ratio 0.5 are discussed), more experimental data needs to be gathered on the different aspect ratios. Still, for the 0.5 aspect ratio the research question 4 has been properly addressed. 

\subsection{Constitutive modelling of thermoplastic amorphous polymers (research question 5)}
%Through an intensive study on the non-linear behaviour of polymers, appropriate stress based yield and strain based failure models were implemented in an explicit analysis where the response to loading conditions in the three principle directions have been investigated. Amorphous thermoplastics have complex mechanical behaviour, as is described in \ref{chp:2}, and are therefore limited. This is, among others, dependant on the temperature and strain rate of the load. For this dissertation quasi-static loading at room temperature was assumed. 
Amorphous thermoplastic polymers have complex nonlinear behaviour. However, the models considered herein have been able to satisfactorily predict it.  A stress based yield criterion implemented for epoxies combined with a strain based failure criterion (Johnson-Cook variation) provided satisfactory results for temperature independent, quasi-static loading. Note that despite thermoplastics and epoxies having different molecular structures (e.g. cross links for epoxy), the continuum models seem to appropriately capture both their behaviors. The implemented hardening curve from the filament represented the polymer behaviour in tension (for a strain lower than 5\%) to a satisfactory degree. Damage evolution was disregarded, as it seemed only significant in thermoplastics at higher strains. 

%n was implemented that was originally used for epoxies combined with The inherent molecular behaviour between thermosets and thermoplastics is fundamentally different due to cross-links, therefore material models should only be copied with consideration. For failure a strain based failure criterion was used, which implements a variation of the Johnson-Cook model. This is also implemented by some polymer material models in LS-DYNA. The failure model is made dependant on the total strain, instead of plastic strain. 
%Damage was not implemented in this simulation, since damage was concluded to only significantly affect the material at large strains. Despite these models being largely implemented experimentally, they do simulate the behaviour of the used ABS quite well. Therefore, to the best of the writers knowledge, the implemented constitutive models simulate the non-linear behaviour of thermoplastics as good as is available, which answers the fifth research question.

\subsection{Additional theory for the prediction of mechanical behaviour of FFF parts (research question 6)}
The Rule of Mixtures has been applied in chapter \ref{chp:6} with the Voigt model in the 1 direction, which predicts the elasticity and fracture stress with a error margin of about 10\%. The Reuss model was applied in the 2 and 3 direction, but it is not accurate due to the inconsistent cross section.
Additionally, Linear Elastic Fracture Mechanics (LEFM) has been applied to predict the fracture stress of different aspect ratios. The cavity has been simplified as a crack in LEFM, and the general trend (qualitatively) agreed with the high-fidelity finite element analyses of the RVEs, although LEFM could not be used for quantitative predictions.

\section{Recommendations}
\begin{itemize}
  \item Simulated RVE results are generated for different aspect ratios, these simulations should be validated with empirical tests of the respective aspect ratios (besides 0.5)).

  \item The simulated RVE results give a representation of fully healed polymers, respective empirical tests should be conducted with fully healed samples. Conversely, if the samples are not fully healed, then the cohesive behavior between roads needs to be characterized and appropriately modeled by cohesive elements.
  
%  \item Elevating the envelope temperature even more can facilitate the investigation of the porosity cavities of the mesostructure on the mechanical properties should be conducted while testing the effect this has on the mechanical properties.
  
  \item The effect of adequate hardware, microtome and filament clamps, should be compared with the methods and results used in this thesis.
  
%  \item After the effect of healing is further investigated with additional high temperature envelope tests, cohesive elements with a lower toughness can be added to simulate sub optimal healing.
  
  \item Different lay-up of layers and orientations should be considered to find the best combination that improves the mechanical behaviour for a given application. 
  
  \item Damage could be implemented based on the craze criterion presented in this thesis. 
  

\end{itemize}

%After optimizing the FEM model, the simulations returned results that had significant overlap with the ASTM D 3039 standard tests in each direction as was discussed before. Implementing damage and permanent plasticity is quite difficult, since amorphous polymers exhibit a non-linear elastic part up to the yield point, where damage starts to occur. The implemented model is coded to start the hardening curve at the anelastic part. Fundamental research in polymer mechanics could in the future come up with more accurate models that predict the mechanical behaviour of amorphous thermoplastics with more accuracy. 
%
%Despite the simulated results already achieving high accuracy, a damage model could be implemented based on crazing criteria and strain rates to simulate polymer behaviour even better. The FEM model could be optimized even further by finishing the mesh convergence study with a set of finer meshes, this is however, due to the computationally expensive procedure, difficult.
%
% The difference in the 3 direction is explained through the sub-optimal healing between roads. This seemed to have less effect on the mechanical properties than first was expected. With the currently used process parameters, the sub-optimal healing caused a decrease of 10\% relative to the bulk UTS, the rest of the decrease in mechanical properties is provoked by the presence of the periodic porosity. To strengthen this statement, more validation is needed for different aspect ratios using the ASTM D 3039 standard (or a variation).  
%
%
%When the RVE model is validated for certain aspect ratios, combined RVE's can be modeled to simulate the behaviour of multiple layers with different directions. Since layers in different orientations might limit stress concentrations and unstable crack growth, a potential optimum might be found. This search could benefit from the use of machine learning, through promoting factors that contribute to optimal mechanical behaviour. Additionally, new materials can be introduced to the RVE, along with composite structures with short or long fibre reinforced material.  
%
%------------------------
%The current literature and efforts in characterizing the mechanical properties of FFF products have been limited to experimental investigations and predictions on elasticity. A lack of knowledge on the process and a large variation in quality between systems lead to a significant divergence in the use of standards, process parameters and research directions. By researching the fundamental questions on the process and the effect on the mechanical behaviour we were able to identify important process parameters, and the causes of the degradation in mechanical properties. FFF products show significant anisotropy in the 3 principal directions, this is due to the lamina like build up of roads that induce bad healing between layers and most importantly, a large amount (almost 3\%) of periodic porosity. The process parameters that largely affect this are: the aspect ratio between the layer height and the line width, the flow multiplier, the layer orientations, the envelope temperature, the deposition speed and the overall accuracy of the system.
%
%In this thesis research was conducted on ABS filament on a Ultimaker S5 system, one of the most advanced systems in its category available to this date. Trough investigating the porosity in the mesostructure of the FFF product generated by the Ultimaker S5, it could be seen that the mesotructure has its differences and similarities in comparison with the results presented in by other researchers. The mesostructure showed a clear periodicity, with triangle like cavities. When the slicing process was investigated further, it became clear that the printing strategy involved overlapping of roads to minimize porosity. This was validated by micrographs of multi-coloured mesostructures. From this analysis, an RVE was developed based on the overlap theory. The geometry of the RVE was made dependent on the line width and layer height determined by the slicer. 
%
%Simultaneously, empirical tensile tests were performed in the 3 principal direction for validation of the RVE simulations. Unfortunately, no testing or production standards were available for testing FFF products, therefore 2 different methods were used, the ISO 527 and ASTM D 3039, both applied in literature. Both standards had their flaws and the results did not indicate a favourite.  
%By using an homogenization approach combined with FEM based on the defined RVE, the elastic properties in the 3 principle directions could be predicted within roughly 5\% accuracy. The Rule of Mixtures predicted the elastic properties with less accuracy in the 2 and 3 direction, since the cross sections are not constant. 
%
%Through an intensive study on the non-linear behaviour of polymers, good stress based yield and strain based failure models were implemented in a explicit analysis where the response to loading conditions in the three principle directions have been investigated. After optimizing the FEM model, the simulations returned results that had significant overlap with the ASTM D 3039 standard tests in each direction, showing a deviation of roughly 5\% in the 1 and 2 directions. Despite the simulated results already achieving high accuracy, a damage model could be implemented based on crazing criteria and strain rates to simulate polymer behaviour even better. The FEM model could be optimized even further by finishing the mesh convergence study with a set of finer meshes, this is however, due to the computationally expensive procedure, difficult. The difference in the 3 direction is explained through the sub-optimal healing between roads. This seemed to have less effect on the mechanical properties than first was expected. With the currently used process parameters, the sub-optimal healing caused a decrease of 10\% relative to the bulk UTS, the rest of the decrease in mechanical properties is provoked by the presence of the periodic porosity. To strengthen this statement, more validation is needed for different aspect ratios using the ASTM D 3039 standard (or a variation).  Subsequently these results can be used to validate the relation with Linear Elastic Fracture Mechanics. If an equation can be fitted for the fracture stress dependent on the the aspect ratio, an analytical approach (which is less time consuming) might suffice for the prediction of the fracture stress. 
%When the RVE model is validated for certain aspect ratios, combined RVE's can be modeled to simulate the behaviour of multiple layers with different directions. Since layers in different orientations might limit stress concentrations and unstable crack growth, a potential optimum might be found. This search could benefit from the use of machine learning, through promoting factors that contribute to optimal mechanical behaviour. Additionally, new materials can be introduced to the RVE, along with composite structures with short or long fibre reinforced material.  
%
%Ultimately, a new element could be defined with the properties of optimal process parameters obtained from the RVE. The properties of this element could be used for general FEM simulations for parts in engineering applications. When these properties are known, even topology optimization can be applied to find the optimal shape for the loading state of the body. 
%When this is achieved, the FFF will truly manifest its potential trough its large geometrical freedom, creating parts on location that perfectly fit the need of the application with minimal material used. 
%
%\todo{By the way: in the text you refer to a few Appendices. Where are they?}